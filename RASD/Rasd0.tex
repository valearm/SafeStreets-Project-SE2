\documentclass[12pt]{article}
\usepackage{mathpazo}
\usepackage{graphicx}
\usepackage[explicit]{titlesec}
\usepackage{hyperref}
\usepackage[export]{adjustbox}
\usepackage{placeins}
\usepackage{subcaption}
\usepackage{import}
\usepackage[dvipsnames]{xcolor}
\usepackage{listings}
\usepackage{setspace}


\begin{document}
\onehalfspacing
%title page
\begin{center}
	{\huge {\textbf{Politecnico di Milano}}}
	 	\vspace{7mm}\\
	 	
 	 	\includegraphics[scale=1.5]{Images/PolimiLogo.png}
	\end{center}

\begin{center}
	     \vspace{5mm}
		{\Large A.A 2019/2020} 
		\vspace{5mm}\\
		{\Large {\textbf{Requirements Analysis and Specifications Document}}}   
		 \includegraphics[scale=1]{Images/LOGO.jpg}
    \end{center}
          
\begin{flushright}
         
	 	 
	 	{\huge {\Large \textbf{Authors:}}}
	 	 
	 	{Armenante Valerio}
	 	 
	 	{Capaldo Marco}
	 	 
	 	{Di Salvo Dario}
	\end{flushright}


%begin of contents

\newpage
\hrule
\tableofcontents %the table of the contents 

\vspace{0.5mm}
\vspace{0.24mm}

\newpage

\section{Introduction} %section introduction
\hrule
\vspace{8mm}
\subsection{Purpose} %subsection purpose
\vspace{5mm}
       The aim of this document is to give an overview of the             requirements and specifications of the system to be developed. The goal of the document is to describe in detail all functional and non-functional requirements of the system, analysing the needs of the customer and explaining common use case scenarios. It will set a baseline for project planning and cost es- timation, giving a detailed insight to all stakeholders which include the SafeStreets Investors and engineers (present and future) involved in development, testing and maintenance. 

\subsection{Scope} %subsection scope
\vspace{5mm}
\subsubsection{Description of the given problem}
\vspace{2mm}
	
	\textbf{SafeStreets Basic Service}

\vspace{3mm}
The given problem is to allow users to notify traffic violations to authorities. More specifically, the notification process consists in the acquisition of pictures, date, time, location and type of reported violation inserted by user. The aim of SafeStreets basic service is to retrieve and store information from inputted data, completing it with suitable metadata.  In particular, when user sends pictures related to a traffic violations, SafeStreets is able to recognize license plate by means of an external algorithm. After that, SafeStreets dispatches the notification to the authority member, which is in an available status and is chosen according to the minimum distance between the notified traffic violation and authority member’s position. In the end, the assigned authority will opportunely manage the received request. In addiction, users and authorities member, through a Mobile Application interface presenting different level of visibility to different roles, will be able to mine information that has been received, for example by highlighting the areas with the highest frequency of violations. This is going to be realized designing some algorithms able to generate statistics on the previously stored data. A user who approaches the mobile application for the first time has to complete a registration process, as well as an authority which approaches the dedicated website for the first time. In particular, an authority inserts all their authority member in SafeStreets database through a website’s form specifying their e-mail. SafeStreets will generate some random credential, and an activating link, sending them to each authority member through an e-mail. From now on, clicking on activating link, authority will be officialy registered to SafeStreets service.

\vspace{5mm}

\begin{flushleft}
\textbf{SafeStreets Advanced Function 1}
\end{flushleft}
\vspace{3mm}
The goal of this SafeStreets’ first advanced function is to improve more the security of cities through a cooperation with municipality. First of all SafeStreets retrieves information from municipality through an external service, then information are categorized (accidents, violations, etc.) and some statistics are generated. Then crossing information between resultant elaboration and already available SafeStreets data, the goal can be reached. More specifically, if traffic violations cardinality, related to an area, exceed a specified threshold, SafeStreets identifies the area as unsafe and suggests possible interventions to municipality. Obviously, the threshold is properly defined according to the characteristics of the city.

\newpage

\begin{flushleft}
\textbf{SafeStreets Advanced Function 2}
\end{flushleft}
\vspace{3mm}
The goal of this SafeStreets’ second advanced function is to provide correct information, related to traffic violations, to municipality(in particular local police). In this way, municipality can easily and correctly emits traffic tickets to the offenders. In order to ensure the correctness of information, SafeStreets implements an algorithm that recognize if a manipulation is occurred on uploaded data by users. Moreover, Safestreets builds some statistics regarding traffic tickets data such as a ranking of the most offenders, most common violations and the effectiveness of SafeStreets initiative
\vspace{4mm}

\subsubsection{Goals} %subsubsection goals


















\end{document}






